% Paper for submission to Pattern Recognition Letters

\documentclass{elsart}   %%[10pt]
%\usepackage[english]{babel}
\usepackage{amssymb}
\usepackage{pclark,epsfig,mysubfigure,comment}

% For revision highlighting
\usepackage[usenames]{color}
\newcommand{\newbyjoey}{
% \color{Red}
}

\begin{document}

\begin{frontmatter}

%%\title{Recognition of Text in 3D Scenes Using Vanishing Points}
\title{Rectifying Perspective Views of Documents in 3D Scenes Using Vanishing Points}

\author{Paul Clark and Majid Mirmehdi}  %%% \corauthref{cor1}}
%\ead{\{pclark,majid\}@cs.bris.ac.uk}
%%\corauth[cor1]{Corresponding author: pclark@cs.bris.ac.uk}

\address{
Department of Computer Science, Univerity of Bristol, Bristol BS8 1UB, UK %\\
%{\tt \{pclark|majid\}@cs.bris.ac.uk}
}

\begin{abstract}
Documents may be captured at any orientation when viewed with a camera.  Here, 
a method of recovering a fronto-parallel view of text documents is presented
in single images.
Analysis of standard 2D projection profiles is commonly used in
document recognition. We introduce a novel extension of this to locate the
horizontal vanishing point  of the text plane.  Following further analysis,
we segment the lines of text  to determine the style of
justification of the paragraphs.  The change in line spacings exhibited due to
perspective is then used to locate the document's vertical vanishing point. 
No knowledge of the camera focal length  is assumed. Using the vanishing
points, a fronto-parallel view is recovered. 
% suitable for OCR or other high-level recognition.  
We provide results demonstrating the algorithm's performance on
documents over a wide range of orientations.
\end{abstract}
%\maketitle

\begin{keyword}
Document Analysis and Recognition \sep Vanishing Point Detection \sep
Perspective Recovery \sep Paragraph Format
\end{keyword}
\end{frontmatter}

%\baselineskip 24pt
\small

\section{Introduction}
With the use of cheap and simple (digital) cameras becoming common-place in the 
home and the office, the demand for simple, instantaneous scanning of documents is
rising. Pointing a camera at a document, clicking a button to capture the image
and then using some software to interpret the text, as in OCR, has many
advantages. For example, it is fast and removes the need for a flatbed scanner, and allows
for non-contact point-and-click capture of text documents. However, difficulties
 arise if the view of the document is perspectively skewed. Then,
some form of rectification is necessary to obtain a fronto-parallel view of the
document plane to allow off-the-shelf OCR software the best chance of
interpretation. 

There has been little research into the recognition of text in real scenes in
which the text is oriented relative to the camera.  As well as for general OCR
document recognition, processing and compensating for perspective skew of text
has applications in assisting the disabled and/or visually impaired, vehicle
navigation (road signs) and control (number plates), wearable computing tasks
requiring knowledge of local text, and general automated tasks requiring the
ability to read where it is not possible to use a scanner. In recent years, we
have presented different methods for locating, segmenting and recovery of 
text in real
scenes \cite{joey1,ClarkICPR2000,2001-clark,justin,clark-ijdar-2001}. As a
result, we developed an approach for {\em estimating} the rectification of the
perspective view of a document \cite{2001-clark,clark-ijdar-2001}. In this work,
we address the issue in more depth and present a {\em robust} approach for
determining the vanishing points of a perspectively viewed document to rectify
the text plane into a fronto-parallel view. Our method is independent of the focal
length of the camera and document font size. Additionally, we analyse and show
the limits and range of orientations of the text documents to which our proposed
method can be applied.


%%% DISCUSS A BY-PRODUCT OF OUR APPROACH AS A PARAGRAPH RECOGNISER 


%\section{Previous work}
There are various works on correction of text documents, when rotated or skewed
in the view plane only, such as \cite{Yu96,Amin96,docsthrucams,messelod1}. 
%% Can add the messeldi work too or remove the docsthrucams
Most such methods correctly use 2D assumptions such as parallel lines in the
view plane to determine the parameters for rotational correction. 
However, other than our own previous work, the authors are aware of only two
other groups involved in perspective skew recovery of documents
\cite{pilucvpr1,dance02}. In \cite{pilucvpr1}, a method is presented on extraction of linear illusory clues in
skewed documents. Horizontal clues  are extracted from
a binarised input image where the characters, words (partial or full) and lines
are transformed into blobs. Association networks are then built based on the
relationships between neighbouring compact or elongated blobs. Using a pool of
horizontal clues, a partial rectification of the document is performed.
Vertical clues are then sought in this new intermediary image to help perform a
full rectification. However, as explained in \cite{pilucvpr1},  the vertical
rectification is dependent on the number of illusary clues obtained as well as
on how reliable they are. In \cite{dance02}, sets of parallel lines
corresponding to text lines and formatted column boundaries are grouped to 
estimate vanishing points which are then used for perspective correction.
The author does not establish how the sets of lines are originally obtained.
This approach provided only an estimation and was  tested on orientation
angles varying only $20^\circ$ between the optical axis of the camera and the
normal to the document. Furthermore, it also only works on fully formatted
(i.e. on both left and right 
side) paragraphs whereas our method will  recognise different types of
justifications in paragraphs of text and work for a much
wider range of orientations. A further important difference between our work and 
those of both \cite{pilucvpr1} and \cite{dance02} is that while their methods
are applied to whole images of a document as the only object in the scene, we have the ability to locate and segment
paragraphs in an image into seperate entities amongst other clutter in the image.
%%% Useful ref: Shu99 (see dance02) 


% \begin{comment}
Outside of the document recognition area there are  many works that
estimate the orientation of planar surfaces in still images 
using  repeating textures or specific object models based on some image features.
For example, Ribeiro and Hancock \cite{ribeiro}
observed affine distortions in power spectra of an image to find lines of
consistent texture indicating the vanishing points of a plane. Criminisi and
Zisserman \cite{zisserman} also used texture to find the vanishing points of a
plane,  first by finding the orientation of the vanishing line with normalised
auto-correlation, and then applying another similarity measure to find its position.
Text too has repetitive elements (characters and lines) but
these elements do not match each other, and also sometimes may cover only a
small area of the image.

%Their approach splits the 2D search into two 1D searches: first normalised
%auto-correlation is used to find the orientation of the vanishing line; they
%then introduce a projective correlation similarity measure to find the position
%of the vanishing line.  We too will adopt the sequential approach of finding one
%dimension of the solution at a time. However,

% Knowledge of the structure of text may guide
% Furthermore the regions we are examining may only fill a relatively small area of
% the plane, and are not of a high resolution.
% Therefore the general approach of using repeating textures to discover
% the orientation of the text does not directly apply to our problem.

Rother \cite{rother0} found vanishing points  corresponding  to the three mutual
orthogonal directions of the scene using the parallel lines commonly
available in architectural environments.
Murino and
Foresti \cite{MURIN} used a 3D Hough transform to estimate the orientation of
planar shapes with known rectilinear features.
Gool et al. \cite{vangool98planar} have employed invariants to recover scene geometry
from image points with known planar homologies.
All of these methods initially require selecting points in the image which are
believed to have a certain relationship in the scene.
In our work, we first detect probable paragraphs and lines of text in the image.
We then attempt to minimise the error between the points and a model constructed from
a-priori information about documents.
The fitting equation detects a set of points that satisfy the constraints
and finds parameters for the 
% However, the problem of choosing a set of points 
% FIX THE GOOL REVIEW
% Gool et al. 
% \cite{GOOL} %and Yip\cite{yip} 
% found the skewed symmetry of 2D shapes which have an axis of
% symmetry in the plane, allowing for affine recovery.  We require recovery from
% perspective transformation and, as with these latter works, we will use a priori
% information about the 2D shape we are observing.

Knowledge of the principal vanishing points of the plane on which text lies is
sufficient to recover a fronto-parallel view. The text lines of a paragraph
on a plane oriented relative to the camera  point towards
the horizontal vanishing point of the text plane in the image. We make the
reasonable assumption that a paragraph must display some sort of
 left,  right, centred or full  formatting, i.e. with straight margins on the
left and/or right, or if the text is centred, a central vertical line around which
the text is aligned.  In such cases, these vertical lines point toward the
vertical vanishing point of the text plane.  We have therefore concentrated our
work on the recovery of paragraphs using these principles to extract the
horizontal and vertical vanishing points. 
%since by inherent nature of paragraphs  some
%formatting and justification exists.  

One approach to building a model of a paragraph is the bottom-up grouping of
pixels into characters, words and lines. This can be a noisy and troublesome
process. Here, we use all of the global information about the paragraph at one
time.  The principle of 2D projection profiles are extended to the problem of
locating the horizontal vanishing point by maximising the separation of the
lines in the paragraph.  From this vanishing point we are then able to
accurately segment paragraphs into separate lines of text.  The segmented lines
are then analysed to reveal the style of justification or alignment of the
paragraph.  Depending on the type of paragraph, either the margins or the projective
spacings between the lines are used to find the vertical vanishing point. For
fully formatted paragraphs the vertical vanishing point is found
using the paragraph's straight margins. For all other paragraphs,  i.e. left,
right, or centrally formatted,  line spacings are analysed to find
the position of the vertical vanishing point accurately. 
The proposed method eliminates the need for knowledge of 
the focal length of the camera, hence the techniques are applicable to images
taken from cameras with unknown internal parameters.

 
 In \refsect{recoveryusinglines}
we briefly review our previous work which provides the input to the work
described here.  Section \ref{locatehvpsect} concentrates on locating the
horizontal vanishing point. This information is applied in Section
\ref{sec-parags} to determine the formatting style of paragraphs.
In Section \ref{sec-vertvanish}  we demonstrate our method of precisely
locating the vertical vanishing point. We evaluate 
the range of orientations our methods apply to for both vanishing points. 
We discuss our conclusions in \refsect{conclusions}.
\section{Finding Text Regions}   \label{recoveryusinglines}

%The methods described in this paper can be applied to images where the main
%object is the perspective view of a text plane and little or no other clutter
%such as those experimented with in both \cite{pilucvpr1} and \cite{dance02}.
%This way the whole image is searched for clues and features regardless of where
%paragraphs of text lie. We can also apply the methods described in
%\cite{ClarkICPR2000,clark-ijdar-2001} where a  a text segmentation algorithm was 
%introduced to segment regions of text in cluttered scenes. This used localised
%texture measures to train a neural network to classify areas of an image as text
%or non-text. \reffig{runblobbed} shows a large region of text which was found in
%\reffig{runorig} using this approach. The benefit here is that operations can be
%carried out on the text region only rather than on the whole image.  In this
%work we consider the output of the system presented in
%\cite{ClarkICPR2000,clark-ijdar-2001} and analyse each region individually 
%to recognise the shape
%of the paragraph, recover the 3D orientation of the text plane, and generate a
%fronto-parallel view of the text. The emphasis here is not on how the text
%region was originally found, but on what can be done next to deal with its
%accurate rectification.


The source images input into the methods proposed in this paper can be typical
PDA-based point-and-click direct images of text documents or regions of text
segmented in more cluttered scenes, e.g. using our earlier work in
\cite{clark-ijdar-2001}. In the simpler point-and-click scenario, 
the whole image can be searched for clues and features knowing of the presence
of a paragraph of text. To deal with cluttered scenes, we described two alternative
algorithms in \cite{clark-ijdar-2001} to segment regions of text first, followed
with a simple approach to recovery of text orientation. In this work, we can
take direct point-and-click iamges from a hand-held camera or 
the output of the segmenation system presented in
\cite{clark-ijdar-2001} and analyse each recognised region individually to
classify the shape of paragraphs, recover the 3D orientation of the text plane
more robustly, and generate a fronto-parallel view of the text. However, the
emphasis here is not on how the text region was originally obtained or
segmented, but on what can be done next to deal with its accurate rectification.


\begin{figure}[h]
\begin{centering}
  % \subfigure[{\small Original Image}]{\epsfig{figure=images/chem002.ps,width=45mm}\label{runorig}}
  \subfigure[{\small Original image}]{\epsfig{figure=images/prlrunning/original.eps,height=33mm}\label{runorig}}
  \hspace*{1mm}
  % \subfigure[{\small Located text regions}]{\frame{\epsfig{figure=images/chem002blob.ps,width=45mm}\label{runblobbed}}}
  % \subfigure[{\small Located regions of interest}]{\frame{\epsfig{figure=images/prlrunning/blobbed.eps,height=37mm}\label{runblobbed}}}
  \subfigure[{\small Classified text region}]{\epsfig{figure=images/prlrunning/blobbed.eps,height=33mm}\label{runblobbed}}
  \hspace*{1mm}
  % \subfigure[{\small After adaptive thresholding}]{\frame{\epsfig{figure=images/chem002bin.ps,width=40mm}\label{runbin}}}
  % \subfigure[{\small One region after adaptive thresholding}]{\frame{\epsfig{figure=images/prlrunning/binarised.eps,height=37mm}\label{runbin}}}
  \subfigure[{\small Text region after adaptive thresholding}]{\epsfig{figure=images/prlrunning/binarised.eps,height=24mm}\label{runbin}}
\label{runprep}
\caption{Preparation of paragraph for planar recovery}
\end{centering}
\end{figure}

In order to analyse the paragraph shape, we first require a classification of
the text and background pixels to obtain a binary representaion.  Dealing with regions of text only, this
classification is simplified through thresholding since the region will contain
easily separable background and foreground colours only. The thresholding does not
need to be too accurate and breakages in characters or words is not detrimental
to later stages. Nevertheless, we try for best results and  compute the average
intensity of the image neighbourhood as an {\em adaptive local threshold}
for each pixel, in
order to compensate for any variation in illumination across a text region.
Partial sums \cite{partialSums} are employed to generate these local thresholds
efficiently.  To ensure the correct labelling of both dark-on-light and
light-on-dark text, the proportion of pixels which fall above and below the
thresholds is considered.  Since in a block of text there is always a larger
area of background than of text elements, the group of pixels with the lower
proportion is labelled as text, and the other group as background.  The example
shown in \reffig{runbin} demonstrates the correct labelling of some light text
on a dark background and is typical of the input into the work presented here.
%% SHOW ITS THRESHOLD HISTOGRAM GRAPH? Surely too obvious anyway.



\section{Locating the Horizontal Vanishing Point}  \label{locatehvpsect}

In \cite{messelod1}, Messelodi and Modena demonstrated a text location method
on a database of images of book covers.  They employed projection profiles to
estimate the skew angle (in the view plane) of the text.
A number of potential angles were found from pairs of components in the text,
and a projection profile was generated for each angle.  They observed that the
projection profile with the minimum entropy corresponded to the correct skew
angle.  
This is to be expected since the
profile at the correct angle will have well-defined peaks
and troughs corresponding to each line of text and the gaps
between them.  Projection profiles at other angles will cause
lines to overlap, merging peaks and troughs, and increasing the
entropy of the profile.
This guided 1D search is not directly applicable to our problem, which is to
find a {\em vanishing point} in $\mathbb{R}^2$, with two degrees of freedom. 
In order to search this space, we will generate projection profiles
from the point of view of vanishing points, rather than from skew angles.


We need a circular search space $C$ as illustrated in \reffig{searchspacediag}.
% although a rectangular grid of polar coordinates would also work effectively.
Each cell $c=(r,\theta)$, $r\in[0,1)$ and $\theta\in[0,2\pi)$, in the space $C$
corresponds to a hypothesised vanishing point $\myvec{V} = (V_r,V_{\theta})$ on
the image plane $\mathbb{R}^2$, with scalar distance $V_{r}=r/{(1-r)}$ from the
centre of the image, and angle $V_{\theta}=\theta$.  This maps the infinite
plane $\mathbb{R}^2$ exponentially into the finite search space $C$.
A projection profile of the text is generated for
every vanishing point in $C$, except those lying within
the text region itself (the central hole in \reffig{ppmap}).


\begin{figure}[h!]
\centering
\begin{center}
\subfigure[Relationship between search space $C$ and $\mathbb{R}^2$]{
\epsfig{figure=images/ppsearchspace.eps,width=45mm}
\label{searchspacediag}
}
\hspace{5mm}
% \subfigure[Scores for all projection profiles in $C$ generated from \reffig{runbin}]{\epsfig{figure=images/chem002ppmap.ps,width=45mm}\label{ppmap}}
\subfigure[Scores for all projection profiles in $C$ generated from \reffig{runbin}]{\epsfig{figure=images/prlrunning/ppmap.eps,width=45mm}\label{ppmap}}
\end{center}
\vspace*{0mm}
\caption{Search space $C$}
\label{searchspace}
\end{figure}

%The projection profile  with respect to a vanishing point
%$\myvec{V}$ is obtained by collecting each pixel into a bin representing the
%angle between the vanishing point and the pixel.  Relative to the vanishing
%point, the text region will fall within a small range of angles, depending on
%the distance and position of the vanishing point.  In order to accumulate
%projection profiles which are comparable to each other, the angular range over
%which each projection profile is taken is set accordingly.  In other words, the
%left and right bins of the projection profile correspond to the left and right
%angles within which the text region lies.  This ensures that the projection
%profile of a distant vanishing point will not accumulate a tighter distribution,
%as would be the case if all projection profiles were accumulated in the range
%$0-2\pi$.
%%*** !Kill above paragraph?  Keep next but make clearer! ***
%%*** !Alternative paragraph to the last!  Which do you prefer?!

A projection profile $B$ is a set of bins $\{B_i, i=0,..,N\}$ into which image
pixels are accumulated.  In the classical 2D case, to generate the projection
profile of a binary image from a particular angle $\phi$, each positive pixel
$\myvec{p}$ is assigned to bin $B_i$, where $i$ is dependent on $\myvec{p}$ and
$\phi$ according to the following equation:

\begin{equation} \label{classicproj} 
i(\myvec{p},\phi) = \frac{1}{2} N + N
\frac{ \myvec{p} \cdot \myvec{U} }{ s } 
\end{equation}

{ \parindent 0mm where $\myvec{U}=(\sin \phi,\cos \phi)$ is a normal vector
describing the angle of the projection profile, and $s>N$ is the diagonal
distance of the image.  In this equation, the dot product $\myvec{p} \cdot
\myvec{U}$ is the position of the pixel along the axis of the projection profile
in the image defined by $\phi$.  Manipulation with $s$ and $N$ is then employed
to map from this axis into the range of the bins of the projection profile.  }



\begin{figure}[t]
\centering
\begin{center}
% \epsfig{figure=images/chem002projprofs.ps,width=85mm}
\subfigure[Generating a projection profile of the text from vanishing point ${\bf V}$.]{
	\begin{tabular}{c}
		\epsfig{figure=images/prlrunning/fan93img.eps,width=70mm}\label{projproffing}
	\end{tabular}
}
\subfigure[The winning projection profile and a poor example]{
	\begin{tabular}{c}
		\epsfig{figure=images/prlrunning/bestpp.eps,width=40mm} \\
		\epsfig{figure=images/prlrunning/badpp.eps,width=40mm}
		\label{projprofs}
	\end{tabular}
}
% \epsfig{figure=images/prlrunning/fan93img.eps,width=40mm}
\end{center}
\vspace*{0mm}
% \caption{Two potential vanishing points $\myvec{V}_A$ and $\myvec{V}_B$, and their projection profiles.}
\caption{Generating projection profiles from \reffig{runbin}}
\label{projprofsfig}
\end{figure}




In our case, instead of an angle $\phi$, we have a point of projection
$\myvec{V}$ on the image plane, which has two degrees of freedom.  Our bins,
rather than representing parallel slices of the image along a particular
direction, must represent angular slices projecting from $\myvec{V}$.  Hence, we
refine (\ref{classicproj}) to map from an image pixel $\myvec{p}$ into a bin
$B_i$ as follows:


\begin{equation} \label{persproj} i( \myvec{p,\myvec{V}} ) = \frac{1}{2} N + N \frac{ \mbox{ang}(\myvec{V},\myvec{V}-\myvec{p}) }{ \Delta \theta } \end{equation}

{ \parindent 0mm 
where $\mbox{ang}(\myvec{V},\myvec{V}-\myvec{p})$ is the angle
between pixel $\myvec{p}$ and the centre of the image, relative to the vanishing
point $\myvec{V}$, and $\Delta \theta$ is the size of the angular range within
which the text is contained, again relative to the vanishing point
$\myvec{V}$. $ \Delta \theta$ is obtained from 
}
%$ \Delta \theta =
%\mbox{ang}(\myvec{V}+\myvec{t},\myvec{V}-\myvec{t}) $

% \begin{equation} \Delta \theta = \mbox{ang}(\myvec{V}+\myvec{t},\myvec{V}-\myvec{t}) \end{equation}
\begin{equation} \Delta \theta = \mbox{ang}(\myvec{T}_{\mbox{\tiny L}},\myvec{T}_{\mbox{\tiny R}}) \end{equation}

{ \parindent 0mm 
% where $\myvec{t}$ is a vector perpendicular to $\myvec{V}$ with
% magnitude equal to the radius of the bounding circle of the text region
where $\myvec{T}_{\mbox{\tiny L}}$ and $\myvec{T}_{\mbox{\tiny R}}$ are the two
points on the bounding circle whose tangents pass through $\myvec{V}$
(shown
in \reffig{projproffing}).  Unlike $s$ in (\ref{classicproj}), it can be seen that
$\Delta \theta$ is dependent on the point of projection $\myvec{V}$.  In fact
$\Delta \theta \rightarrow 0$ as $\myvec{V}_r \rightarrow \infty$ since more
distant vanishing points view the text region through a smaller angular range.
% The use of $\myvec{t}$ to find $\Delta\theta$ ensures that the angular range
The use of $\myvec{T}_{\mbox{\tiny L}}$ and $\myvec{T}_{\mbox{\tiny R}}$ to find $\Delta\theta$ ensures that the angular range
over which the text region is being analysed is as closely focused on the text
as possible, without allowing any of the text pixels to fall outside the range
of the projection profile's bins.  This is vital in order for the generated
profiles to be comparable, and also beneficial computationally since no bins
need to be generated for the angular range $2 {\pi}-\Delta \theta$ which is
absent of text.
}

\begin{comment}
Relative to a vanishing point $\myvec{V}$, the text region under examination
will fall entirely within a range of angles, forming an arc or wedge shape
extending from $\myvec{V}$.  The two angles which form this enclosing wedge are
used as the left and right bounds of the projection profile.  Hence all pixels
in the text region will map correctly to a bin in the projection profile, whilst
ensuring that the text fully spans the projection and a useful profile is
obtained.  Without this resizing of the window over which to collect the
projection profile for each vanishing point, the profiles obtained will not
relate to each other.  This is because more distant vanishing points find the
text lying within a smaller angular range, which will produce a drastically
different projection unless we focus on the relevant range.
\end{comment}

% For example, if we were to use a large window of $0-2\pi$ to generate all the
%projection profiles, those from the more distant vanishing points would find the
%projection congregating in fewer and fewer bins! 

% We have experimented with taking the entropy, squared-sum, and
% derivative-squared-sum of the projection profile, where:

% \begin{equation} \mbox{Entropy }E(B) = \sum_{i=1}^{N}{B_{i} log(B_{i})} \end{equation}
% \begin{equation} S(B) = \sum_{i=1}^{N}{{B_{i}}^{2}} \end{equation}
% \begin{equation} \mbox{Derivative-squared-sum }DSS(B) = \sum_{i=1}^{N-1}{{(B_{i+1}-B_{i}})^{2}} \end{equation}

% We found $SS(B)$ and $DSS(B)$ always respond accurately
% on a text only region, but in the presence of noisy data the two measures
% perform differently.
% % The entropy measure $E(B)$ performed significantly worse than the other two.
% In our experiments, and in \reffig{ppmap}, the measure we used was the squared-sum, $SS(B)$.
% % The introduction of different types of noise produced different
% % results from each, but we shall not go into that here.
% % although the derivatives can be more informative in the presence of
% % types of noise.
% We note that the accurate calculation of derivatives requires
% the generation of a high-resolution projection profile with proper modelling of the
% pixel contributions.
% Since the sum of squares does not require such accurate data, for this measure
% projection profiles may be generated far more computationally efficiently
% with a lower resolution of bins and a simplistic model
% of pixel contribution.

Having accumulated projection profiles for all the hypothesised vanishing points
using (\ref{persproj}), a simple measure of confidence should be
applied to each projection profile $B$ to mark its significance.
The confidence measure must be chosen to respond favourably to projection profiles
with distinct peaks and troughs.
Since straight lines are most clearly distinguishable from the point where they
intersect, this horizontal vanishing point and its neighbourhood should be
favoured by the measure. We experimented with taking the entropy, squared-sum, and
derivative-squared-sum of the projection profiles, where:
\begin{equation} \mbox{Entropy: } \hspace*{5mm} E(B) = \sum_{i=1}^{N}{B_{i} log(B_{i})} \end{equation}
\begin{equation} \mbox{Squared-sum: } \hspace*{5mm} S(B) = \sum_{i=1}^{N}{{B_{i}}^{2}} \end{equation}
\begin{equation} \mbox{Derivative-squared-sum: } \hspace*{5mm} S_d(B) = \sum_{i=1}^{N-1}{{(B_{i+1}-B_{i}})^{2}} \end{equation}

The derivative measure $S_d(B)$ proved far more resilient to noise than the
other measures, which were easily mislead to view narrow paragraphs from the
top or bottom, rather than the side.
This made $S_d(B)$ the only viable measure for the hierarchical scan or noisy
images.

\begin{comment}
The entropy measure $E(B)$ performed significantly worse in all the above
measures.  $S(B)$ and $S_d(B)$ always responded well with the $S(B)$ measure
deteriorating rapidly as the image became noisier. CHECK JOEY.  $S_d(B)$
responded best even in presence of noise and is also more efficient to compute.
The confidence of each of the vanishing points with regard to the binarised text
in \reffig{runbin} is plotted in \reffig{ppmap}, where darker pixels represent a
larger squared-sum, and a more likely vanishing point.
\end{comment}


\begin{figure}[t!]
\centering
\begin{center}
\subfigure[Fully justified]{\epsfig{figure=images/simulated-full.ps,height=36mm}\label{simulated-full}} 
% \subfigure[Left justified]{\epsfig{figure=images/simulated-left.ps,width=41mm}\label{simulated-left}} 
\subfigure[Centrally justified]{\epsfig{figure=images/simulated-cen.ps,height=36mm}\label{simulated-left}} 
\subfigure[Left justified with extranneous elements]
{\epsfig{figure=images/simulated-left-more.ps,height=36mm}\label{simulated-left-more}} 
\end{center}
\vspace*{0mm}
\caption{Examples of simulated images used for performance analysis.}
\label{simimages}
\end{figure}

To locate the vanishing point accurately, the resolution of the
search space must be sufficient to hypothesise a large number of potential
vanishing points.  During experiments we found empirically that $10^4$
vanishing points was reasonable. Since each vanishing point examined requires
the generation and analysis of a projection profile, a full search of the
space, as shown in \reffig{ppmap}, is computationally expensive.  However, due
to the large scale features of the search space, we introduced an efficient
hierarchical approach to the search.  An initial scan of the search space at a
low resolution is performed, requiring the generation of only a few hundred
projection profiles.  Adaptive thresholding is then applied to the confidence
measures of these projection profiles, to extract the most interesting regions
of the search space.  (In our experiments, this was taken to be the top scoring
2\% of the space.) A full resolution scan is then performed on these
interesting regions in the search space, requiring the generation of a few
further projection profiles close to the expected solution.  Finally, the
projection profile with the largest confidence is chosen as the horizontal
vanishing point of the text plane.  
This hierarchical search reduced the processing time on an HP-UX from over 40
seconds to about 3 seconds. In all our experiments, this adapted search found
the same horizontal vanishing point as when we performed a full search.
The winning projection profile and an
example of a poor projection profile are shown in \reffig{projprofs}, and
marked in \reffig{ppmap} with a white cross and a black cross respectively.
% The hierarchical search reduces the processing time on a Sun Enterprise from
% over two minutes to under ten seconds.

% It is worth noting that for this adapted algorithm, again the derivative measure
% $S_d(B)$ performs far more efficiently than the sqaured-sum and entropy
% measures, in that they can mislead the hierarchical search by also responding
% favourably to the {\em vertical} vanishing point of a `thin' document, i.e. a
% document which has been rotated about its vertical axes.

% JOEY- Can we say something more about noise similar to this commented 
% paragraph I found in your tex file? basically we need to state a little more
% clearly about the reaction of our confidence measure to noise.
% Whilst different types of noise can create different problems,
% in general noisy pixels affect all of the projection profiles similarly,
% and the horizontal vanishing point still retains the greatest confidence.
% Despite general image noise and the resolution of the search space, this method 
% has consistently provided a good estimate of the 
% horizontal vanishing point in our experiments.
% Note that if the best projection profile is found on the edge of the unit
% circle, then the vanishing point is effectively at an infinite distance, and the
% horizontal axis of the text plane is parallel with the image plane.

In order to assess the performance of the algorithm, simulated images such as
those in \reffig{simimages} were generated at various orientations ranging from
$0^\circ$ to $90^\circ$ in both yaw and pitch, resulting in 900 test images. 
\reffig{hvpaccuracy} shows the accuracy of recovery of the horizontal vanishing
point (HVP) for these images, 
calculated as the relative distance of the located vanishing point $\myvec{H}$ 
from the ground truth $\myvec{H}_{\mbox{\small gt}}$, given by

\begin{equation} 
\mbox{HVP accuracy =} -\frac{|\myvec{H}-\myvec{H}_{\mbox{\small gt}}|}{|\myvec{H}_{\mbox{\small gt}}|}
\end{equation}

As can be seen, the proposed method achieves excellent accuracy across a wide
range of orientations. However, not unexpectedly, the performance begins to
drop as the orientation of the plane approaches $90^\circ$ in yaw or pitch. In
these cases, the document has been rotated so as to be almost orthogonal to the
view plane, and hence nearly invisible in the image, explaining the reduction
in performance. The slope of the graph at low yaw may be attributed to the
discretisation of the search space $C$. Since the vanishing points in these
situations lie close to infinity, the distances of the located vanishing points
can not be precise. Nevertheless, the vanishing point chosen will be in the
correct direction, and suitably large so as not to affect further processing.

To determine the affect of noise on the performance, we performed the same test
repeatedly on images with increasing amounts of speckled noise, and averaged the
results over the different orientations.  \reffig{noiseFullVHier} shows the
average accuracy of recovery of the horizontal vanishing point against image noise.
The results when using a full scan of the search space are also plotted, to
demonstrate that there is no loss of accuracy when using the hierarchical scan.
A numerical analysis of the performance is given in \reftab{accuracytable} and
discussed at the end of this paper.


\begin{figure}[t]
\centering
\begin{center}
% \epsfig{figure=images/accuracy-hvp-jagged,width=60mm,angle=-90}
% \end{center}
% \vspace*{0mm}
% \caption{Accuracy of recovery of the horizontal vanishing point for simulated
% paragraphs at various orientations. SHOULD COMBINE WITH NEXT FIGURE}
% \label{hvpaccuracy}
% \end{figure}
\subfigure[
	Accuracy of recovery of a simulated paragraph at various orientations.
]{
	\epsfig{figure=images/accuracy-hvp-jagged.eps,width=40mm,angle=-90}
	\label{hvpaccuracy}
} 
\hspace*{4mm}
\subfigure[
	Accuracy averaged over all orientations plotted against
	increasing image noise.
]{
	\epsfig{figure=images/accuracy-against-noise-on-full-and-hier.eps,height=40mm}
	\label{noiseFullVHier}
} 
% \begin{figure}[t]
% \centering
% \begin{center}
% \epsfig{figure=images/accuracy-against-noise-on-full-and-hier.eps,width=80mm}
\end{center}
% \vspace*{0mm}
% \caption{Average accuracy of recovery of the horizontal vanishing point for
% simulated paragraphs against increasing image noise.}
% \label{noiseFullVHier}
\caption{Accuracy of recovery of the horizontal vanishing point.}
\end{figure}









\section{Determining the Format  of Paragraphs} \label{sec-parags}

The location of the horizontal vanishing point, and the projection profile of
the text from that position, make it possible to separate the individual lines
of text.  This will allow the style of formatting or justification of the
paragraphs to be determined, and lead to the location of the vertical vanishing
point as shall be described in \refsect{sec-vertvanish}.

We apply a simple algorithm to the winning projection profile to segment the
lines.  A {\em peak} is defined to be any range of angles over which all the
projection profile's bins register more than $K$ pixels, with $K$ as the
average height of the interesting part of the projection profile:

\begin{equation}
K= \frac{1}{y-x+1} \sum_{i=x}^{y}B_i
\end{equation}

{\parindent 0mm
where $x$ and $y$ are the indices of the first and last non-empty bins respectively.
A {\em trough} is defined to be the range of angles between one peak and the next.  
The central angle of each trough is used to indicate the separating boundary of
two adjacent lines in the paragraph.  We project segmenting lines from the
vanishing point through each of these central angles.
% in the range.
All pixels in the binary image lying between two adjacent segmenting
lines are collected together as one line of text. Example results of this
segmentation are shown in \reffig{linesegfig}.
% Most lines of text are segmented accurately, although in
% \reffig{chem002overlay} a very short line has been ignored.
Noisy pixels, very short lines, and extraneous document
elements may become attached to a true text line, or be segmented as a separate
line.  However, the processing which follows will compensate for this irrelevant
data. 
} 


\begin{figure}[t]
\centering
\begin{center}
% \subfigure[A fully formatted document]{\epsfig{figure=images/chem002overlay.ps,width=64mm}\label{chem002overlay}\label{summaryfiga}}
% \subfigure[A left-justified paragraph]{\epsfig{figure=images/prlrunning/origover.eps,width=64mm}\label{chem002overlay}\label{summaryfiga}}
% \subfigure[Two centrally aligned paragraphs]{\epsfig{figure=images/prlall/chem015/overboth.eps,width=64mm}\label{summaryfiga}}
% \subfigure[Centralised paragraphs]{\epsfig{figure=images/prlrunning/origover.eps,width=64mm}\label{chem002overlay}\label{summaryfiga}}
\subfigure[The running example.]{\epsfig{figure=images/prlrunning/origover.eps,width=64mm}\label{chem002overlay}}
\hspace{2mm}
% \subfigure[A centrally aligned document]{\epsfig{figure=images/prlall/chem017/origover.eps,width=64mm}\label{chem010overlay}}
\subfigure[Two centrally aligned paragraphs]{\epsfig{figure=images/prlall/chem015/overboth.eps,width=64mm}\label{chem015overlay}}
% \subfigure[Partially left-justified paragraph]{\epsfig{figure=images/prlall/chem005/origover.eps,width=64mm}\label{chem005overlay}}
% \subfigure[]{\epsfig{figure=images/chem005001origover.ps,width=58mm}\label{chem005001overlay}}
\end{center}
\vspace*{0mm}
\caption{Example of recognition: 
Line segmentations are marked in white, points for margin fitting in green
(used) and red (rejected outliers), the baseline in yellow,
the rectangular frame on the text plane in blue.}
\label{linesegfig}
\label{summaryfig}
\end{figure}



Depending on the formatting of the paragraph being recovered, there are now two
possible ways to analyse the segmented lines to reveal the vertical vanishing
point.  If the paragraph is {\em fully-justified}, then the left and right
margins of the text are straight, and intersecting these two margin lines will
provide us with the vertical vanishing point, and the problem is fully resolved.

Alternatively, if the paragraph is {\em left-justified}, {\em right-justified},
or {\em centred}, a straight line will be visible either on the left margin, on
the right margin, or through the centres of the lines, respectively.  The
vanishing point will lie somewhere along this {\em baseline} (see Figure \ref{linesegfig}).  However, its actual
position will still be unknown, and must be estimated.  Initially, we must
determine the structure of the paragraph, i.e. its formatting style and its baseline. We collect
the left end, the centroid, and the right end of each of the segmented paragraph lines, to
form three sets of points $P_L,P_C,P_R$ respectively.  Since some justification
or formatting is anticipated in the text document, 
we will expect a straight line to
fit well through at least one of these sets of points, representing the left or
right margin, or the centre line of the text.  To establish the line of
best fit for each set of points, we use a RANSAC (random sampling consensus,
\cite{bolles81ransac-based}) algorithm to reject outliers caused for example by
short lines, equations or headings.  Given a set of points $P$, the line of best
fit through a potential fit $F=\{\myvec{p}_i, i=1,..,M\}\subseteq P$ passes
through $\myvec{c}$, the average of the points, at an angle $\psi$ found by
minimising the following error function:

\begin{equation}
% E_F(\myvec{c},\myvec{n}) = \frac{1}{M^5} \sum_{i=1}^{M} ( (\myvec{p}_i-\myvec{c}) \cdot \myvec{n})^2
E_F(\psi) = \frac{1}{M^5} \sum_{i=1}^{M} ( (\myvec{p}_i-\myvec{c}) \cdot \myvec{n})^2
\label{ransacerroreqn}
\end{equation}

{\parindent 0mm
where $\myvec{n}=(-\sin \psi,\cos \psi)$ is the normal to the line, $M^{2}$
normalises the sum, and a further $M^{3}$ rewards the fit for using a large
number of points. Hence, for the three sets of points $P_L,P_C,P_R$ we obtain
three lines of best fit $F_L,F_C,F_R$ with respective errors
$E_L,E_C,E_R$.  It is now possible to classify the formatting style of the
paragraph using the rules in \reftab{typeofparatable}.
\reffig{chem002overlay} shows the line $F_L$ passing through the left margin of
the paragraph.  In this case $E_L<E_C$ and $E_L<E_R$, hence the second
condition in \reftab{typeofparatable} is satisfied and the paragraph is
correctly identified as being left-justified. The examples in 
\reffig{chem015overlay} show the detection of centrally-justified paragraphs
in a book of poems.
% \reffig{summaryfiga} shows the line $F_C$ passing through the centre of
% the paragraph.  In this case $E_C<E_L$ and $E_C<E_R$, hence the last
% condition in \reftab{typeofparatable} is satisfied and the paragraph is
% correctly identified as being centrally-justified.
}

\begin{table}[t]
  \begin{center}
    \begin{tabular}{|c|c|}
      \hline
      {\bf Condition} & {\bf Type of paragraph} \\
      \hline \hline
      % All three lines have low and similar error:
      $E_L \simeq E_C \simeq E_R$ & Fully-justified. \\
      \hline
%      $E_L<E_R$ and $E_L<E_C$ & Left-justified. \\
      $\min(E_L,E_C,E_R)=E_L$ & Left-justified. \\
      \hline
%      $E_R<E_L$ and $E_R<E_C$ & Right-justified. \\
      $\min(E_L,E_C,E_R)=E_R$ & Right-justified. \\
      \hline
%      $E_C<E_L$ and $E_C<E_R$ & Centrally-aligned. \\
      $\min(E_L,E_C,E_R)=E_C$ & Centrally-justified. \\
      \hline
    \end{tabular}
  \end{center}
  % \vspace*{-5mm}
  \caption{Classifying the type of paragraph given the error fit in (\ref{ransacerroreqn})}
%  \vspace{-5mm} % This is a dirty hack to help page flow
  \label{typeofparatable}
\end{table}



As mentioned earlier, for fully-justified paragraphs, the recovery of the
vertical vanishing point is trivial. This may be achieved by intersecting the
left and right margins of the paragraphs, e.g. $F_L$ and $F_R$, the results of 
which are shown later in \reftab{accuracytable} and \reffig{vvpaccuracya}. 
However, for a left-justified, right-justified or centralised paragraph, we can
retrieve only one baseline.  The other two fitted lines will have significant
errors due to the jagged margin(s).  In these situations, a different method
must be used to determine the position on the baseline at which the vanishing
point lies.








\section{Locating the Vertical Vanishing Point} \label{sec-vertvanish}


In a perspective, non-fronto-parallel view, the spacings between adjacent lines
of text in an image will vary relative to their distance from the camera.  These
appear closer together as they approach the horizon, much like the sleepers
beneath railway tracks. This change in spacing can be used to determine the angle
at which the document is tilted, and hence the vertical vanishing point of the
text plane.

By rotating the image plane to place the baseline vertically, we may disregard the $x$-coordinates and deal solely in the $y,z$ plane, as shown in \reffig{zyspacings}.
Here, the bottom of the paragraph is positioned at $\myvec{P}$ with lines occurring at even spacings of distance $\myvec{Q}$.  Therefore the $n$th line from the bottom of the paragraph will appear at:

\begin{figure}[t]
\centering
\begin{center}
  \epsfig{figure=kkkzyspacings7.eps,width=85mm} %% JOEY, copy this file to your images directory to overwrite the original
\caption{The geometry involved in line spacings.}
\label{zyspacings}
\end{center}
\end{figure}

\begin{eqnarray}
\myvec{L}(n) = \myvec{P}+n\myvec{Q} \label{nthline}
\end{eqnarray}

{\parindent 0mm
Now using simple perspective projection ratios, we have: 
%and will project to the point in the image plane
}
%\begin{eqnarray}
%y(n) = f \frac{ \myvec{L}(n)_y }{ \myvec{L}(n)_z } = f \frac{ \myvec{P}_y + n \myvec{Q}_y }{ \myvec{P}_z + n\myvec{Q}_z } \label{spacingspq}
%\end{eqnarray}

\begin{eqnarray}
\frac{y(n)}{f}  = \frac{ \myvec{L}(n)_y }{ \myvec{L}(n)_z }  \label{spacingspq1}
\end{eqnarray}
{\parindent 0mm
where $f$ is the focal length of the camera. Hence, the perspective projection
of the  $n$th line in the image plane after replacing from (\ref{nthline}) into
(\ref{spacingspq1}) is: 
}
\begin{eqnarray}
y(n) =  f \frac{ \myvec{P}_y + n \myvec{Q}_y }{ \myvec{P}_z + n\myvec{Q}_z } \label{spacingspq}
\end{eqnarray}

{\parindent 0mm
Without losing the nature of the projection, we may scale the scene about the focal point in order to set $\myvec{P}_z$ to $f$, hence modelling the paragraph as if it touched the image plane.  In this case, $\myvec{P}_y=y(0)$,
and we may rewrite \refeqn{spacingspq} as:
}

\begin{eqnarray}
y(n) = U \frac{ 1 + nV }{ 1 + nW } \label{spacingsvweqn}
\end{eqnarray}

{ \parindent 0mm
with $U=y(0)$ and only two unknowns,
$V = {\myvec{Q}_y}/{\myvec{P}_y}$ and $W = {\myvec{Q}_z}/{\myvec{P}_z}$.
The cancelling of the focal length $f$ in this way means that the technique is
applicable to images captured with any optical camera and the internal
parameters of the original camera need not be known.
By projecting the centroids of the lines of text located in the image from the horizontal vanishing point onto the baseline, estimates for $y(n)$ may be obtained.
% for these points.
However, since it is common for documents to also contain lines of text which are not part of an evenly spaced paragraph, and for extraneous elements to enter the data, the $n$th line found in the image may lose correspondence with the $n$th line in the paragraph model.
To fit a curve of position $y(n)$ against line number $n$ would be unwise in this situation.
It is therefore preferable to fit the curve of {\em line spacing} $Y_n$
against {\em position} $X_n$, defined as:

\begin{eqnarray}
Y_n=y(n+1)-y(n) \label{linespacingsdefneqn} \\
X_n=y(n) \label{linepositiondefneqn}
\end{eqnarray}

In this case any odd lines will appear as isolated outliers in line spacing, but will not propagate through the remaining points.
By substituing \refeqn{spacingsvweqn} into the definition of line spacing \refeqn{linespacingsdefneqn}, the curve of $Y$ in terms of $X$ may be written in two parts:
}

\begin{equation}
Y(X) = U ( \frac{1+(n(X)+1)V}{1+(n(X)+1)W} - \frac{1+n(X)V}{1+n(X)W} )
\label{fittingeqn}
\end{equation}

{\parindent 0mm
with $n(X)$ derived by a similar substitution of \refeqn{spacingsvweqn} into the definition of line position \refeqn{linepositiondefneqn} and rearranging to:
}

\begin{equation}
n(X) = \frac{X-U}{XW-UV}
\label{nfromxeqn}
\end{equation}

Initial parameters $V$ and $W$ are chosen for line fitting using a simple estimate for the error optimisation.
However, due to the complexity of \ref{fittingeqn} and \ref{nfromxeqn}, many false minima exist, and one of these may be converged upon during optimisation.
Therefore, to refine the parameters, an initial fit is made with an approximation of \refeqn{fittingeqn}:

\begin{equation}
Y(X) = \frac{ UV }{ 1+n(X)W }
\end{equation}

This ensures that parameters close to the desired minima are obtained before the final fitting.  Once optimised, $V$ and $W$ are plugged into \refeqn{spacingsvweqn} to find the altitude of the horizon $y(\infty) = UV/W$.
By reversing the rotation made earlier to bring the baseline upright, this point will correspond to the location of the vertical vanishing point in the original image.
% We therefore fit the measured positions against spacing $(y(n),y(n+1)-y(n))$ to the model:

\begin{figure}[t]
\begin{centering}
  \subfigure[VVP recovery on fully justified paragraphs by intersecting margins]{\epsfig{figure=images/accuracy-vvp-margins,width=40mm}\label{vvpaccuracya}}
	\hspace{2mm}
  \subfigure[Margin intersection fails on left justified paragraphs]{\epsfig{figure=images/accuracy-vvp-margins-jagged.ps,width=40mm}\label{vvpaccuracyb}}
  \subfigure[VVP recovery on left justified paragraphs by fitting line spacings]{\epsfig{figure=images/accuracy-vvp-spacings-jagged.ps,width=40mm}\label{vvpaccuracyc}}
\caption{Accuracy of recovery of vertical vanishing point (VVP) on simulated paragraphs at various orientations.}
\label{vvpaccuracy}
\end{centering}
\end{figure}

\reffig{vvpaccuracy} shows the accuracy of recovery of the vertical vanishing
point for the whole range of $0$ to $90^\circ$ in yaw and pitch using the
methods described.  In \reffig{vvpaccuracya} it can be seen that, as expected,
intersecting the left and right margins of a fully justified paragraph gives a
good estimate of the vertical vanishing point.  Also as expected and can be
observed in \reffig{vvpaccuracyb}, when margins are used to estimate the
vertical vanishing point of a non-fully formatted paragraph (in this example a
left-justified one), performance is poor due to the paragraph's jagged edge.
%As with the horizontal vanishing point, the method is not suited to situations
%when the vertical vanishing point is close to infinity, which occurs when the
%pitch is low.
Finally, \reffig{vvpaccuracyc} shows an example of the accuracy when the line
spacings are employed on left-justified paragraphs. This method provides good
results comparable to the first for all of the simulated images except those
documents oriented beyond $80^\circ$ in pitch, where the algorithm begins to
fail.
% The method provides comparable results over most of the simulated images, however for documents beyond $80^\circ$ in pitch the algorithm begins to fail.
As with the horizontal vanishing point in \refsect{locatehvpsect}, this may be
explained by the orientation of the document becoming nearly perpendicular to
the image plane. At such an extreme tilt, even if the lines of text are
separated correctly, their proximity in the image means there is little accuracy
in position and spacing for the curve fitting. In real world images, documents
at such extreme angles cannot practically be read or used by OCR once recovered,
hence this failure is not a great concern. The advantage of the line spacings
method is that it provides consistent results for paragraphs which are not fully
justified.
% In contrast, the poor performance of the margins method when dealing with documents which are not fully justified can be seen in \reffig{vvpaccuracyc}.

The results for these experiments, and the location of the horizontal vanishing
point in \refsect{locatehvpsect}, are shown numerically in
\reftab{accuracytable}. 
The vanishing point (VP) error is calculated as the relative distance of the
vanishing point from its ground truth, as described in \refsect{locatehvpsect}. 
The angular error is derived from the final determined orientation of the
horizontal and vertical vectors of the text plane. 
It can be seen that the accuracy of location of the vertical vanishing point in
reasonable for both the margin intersection and the line spacings method. 
As the last row of \reftab{accuracytable} shows, intersecting margins is not
suitable for documents with jagged edges. 

\begin{table}[t]
  \begin{center}
    % \begin{tabular}{|p{2.5in}|l|l|}
    \begin{tabular}{|p{95mm}|l|l|}
      \hline
      {\bf Paragraph (pgh) types } & {\bf VP} & {\bf Angular} \\
      {\bf } & {\bf error} & {\bf error} \\  \hline \hline
      HVP using projection profiles & 0.129 & 2.16$^\circ$ \\  \hline
      VVP using margin intersection for fully formatted pghs & 0.0785 & 2.08$^\circ$ \\ \hline
      VVP using line spacings on what? left? & 0.133 & 3.30$^\circ$ \\ \hline
      VVP using margin intersection on left formatted pghs & 1.23 & 24.50$^\circ$ \\   \hline
      VVP using margin intersection on right formatted pghs & X &  XX$^\circ$ \\ \hline
      VVP using margin intersection on centre formatted pghs & X  &  XX$^\circ$ \\ \hline  \hline
   \end{tabular}
  \end{center}
  % \vspace*{-5mm}
  \caption{Average error for the various methods over $10^\circ$ to $80^{\circ}$ in yaw and pitch.}
  \label{accuracytable}
\end{table}


%% Joey's original table
%\begin{table}[t]
%  \begin{center}
%    % \begin{tabular}{|p{2.5in}|l|l|}
%    \begin{tabular}{|p{8cm}|r@{}l|r@{}l|}
%      \hline
%      {\bf Paragraph (pgh) types } & \multicolumn{2}{c|}{\bf VP} & \multicolumn{2}{c|}{\bf Angular} \\
%      {\bf } & \multicolumn{2}{c|}{\bf error} & \multicolumn{2}{c|}{\bf error} \\  \hline \hline
%      HVP using projection profiles & ~~~0&.129 & ~~~2&.16$^\circ$ \\  \hline
%      VVP using margin intersection for fully formatted pghs & 0&.0785 & 2&.08$^\circ$ \\ \hline
%      VVP using line spacings on what? left? & 0&.133 & 3&.30$^\circ$ \\ \hline
%      VVP using margin intersection on left formatted pghs & 1&.23 & 24&.5$^\circ$ \\   \hline
%      VVP using margin intersection on right formatted pghs & X & &  & XX$^\circ$ \\ \hline
%      VVP using margin intersection on centre formatted pghs & X & & &  XX$^\circ$ \\ \hline  \hline
%   \end{tabular}
%  \end{center}
%  % \vspace*{-5mm}
%  \caption{Average error for the various methods over $10^\circ$ to $80^{\circ}$ in yaw and pitch.}
%  \label{accuracytable}
%\end{table}













Having found the vanishing points of the plane, we may project two lines from each
to describe the left and right margins and the top and bottom limits of the paragraph.
These lines are intersected to form a quadrilateral enclosing the text,
as shown in \reffig{summaryfig}. % , which is expanded to frame the paragraph.
This quadrilateral is then used to recover a fronto-parallel viewpoint
of the paragraph of text.


\begin{figure}[t]
\centering
\begin{center}
% \epsfig{figure=images/image29001recover.ps,width=60mm}
% \epsfig{figure=images/chem002001recover.ps,height=54mm}
% \hspace{2mm}
% \epsfig{figure=images/chem010recover.ps,height=54mm}
\subfigure[Recovery of running example]{\epsfig{figure=images/chem002001recover.ps,height=52mm}}
\hspace{2mm}
\subfigure[Suitable for OCR]{\epsfig{figure=images/chem010recover.ps,height=52mm}\label{arthurcclarke}}
\end{center}
%\vspace{-3mm}
\caption{Fronto-parallel recovery of example documents in \reffig{summaryfig}.}
\label{pprecover}
\end{figure}




\begin{figure}[t]
\centering
\begin{center}

    \subfigure[]{ \begin{tabular}{c}
      \epsfig{figure=images/chem011.ps,width=33mm}\label{banana} \\
      $\downarrow$ \\
           \\
      \epsfig{figure=images/chem011recover.ps,width=26mm}
        \end{tabular}       \label{emppra}
    }
	 \hspace{2mm}
    \subfigure[]{ \begin{tabular}{c}
      \epsfig{figure=images/chem015.ps,width=36mm} \\
      $\downarrow$ \\
           \\
      \epsfig{figure=images/chem015recover.ps,width=36mm}
        \end{tabular}       \label{empprb}
    }
    \subfigure[]{ \begin{tabular}{c}
      \epsfig{figure=images/dv5.ps,width=36mm}
		\hspace{11mm}
      \put(10,38){$\rightarrow$}
		\hspace{11mm}
      \epsfig{figure=images/dv5recover.ps,width=24mm}
        \end{tabular}      \label{empprc}
     }
    % \subfigure[]{ \begin{tabular}{c}
      % \epsfig{figure=images/dv5.ps,width=36mm} \\
      % $\downarrow$ \\
           % \\
      % \epsfig{figure=images/dv5recover.ps,width=36mm}
        % \end{tabular}      \label{empprc}
     % }
% \\ \vspace{5mm}
    % \subfigure[]{ \begin{tabular}{c}
      % \includegraphics[width=35mm]{images/chem006oo.ps} \\
      % $\downarrow$ \\
          % \\
      % \includegraphics[width=35mm]{images/chem006002recover.ps}
        % \end{tabular}      \label{empprd}
     % }
    % \subfigure[]{ \begin{tabular}{c}
        % \includegraphics[width=35mm]{images/chem008oo.ps}   \\
      % $\downarrow$ \\
          % \\
        % \includegraphics[width=35mm]{images/chem008001recover.ps}   \\
        % \includegraphics[width=35mm]{images/chem008002recover.ps}
    % \end{tabular} \label{emppre} }
    % \subfigure[]{ \begin{tabular}{c}
        % \includegraphics[width=35mm]{images/clut4allover.ps}       \\
      % $\downarrow$ \\
           % \\
        % \includegraphics[width=35mm]{images/clut4recover1.ps}    \\
        % \includegraphics[width=35mm]{images/clut4recover2.ps}
    % \end{tabular} \label{empprf} }

\end{center}
\vspace*{-3mm}
\caption{Further examples of fronto-parallel recovery of documents. In each case the original image is shown above the recovered output.}
\label{againmoreppresults}
\end{figure}

%% JOEY USE FIG6b as a new example in Figure 10.
\reffig{pprecover} shows the rectified images of the examples in
\reffig{linesegfig}. 
Further examples in \reffig{againmoreppresults}
show the recovery of different styles of paragraphs with left justified and
centrally aligned text.
\reffig{emppra} shows the recovery of a segmented region of a book cover.  Despite 
text of different sizes, and other image noise such as the specularity, the document is recovered robustly.
\reffig{empprb} shows a centrally justified paragraph which has been recovered at high resolution and is easily readable.
In \reffig{empprc} a left justified document was correctly identified and recovered, despite occlusion of part of the paragraph.
When we applied commercial OCR software to the image in \reffig{arthurcclarke}, 87\% of the characters and 94\% of the words were recognised correctly.



\section{Discussion} \label{conclusions}

We have presented a novel approach to the fronto-parallel recovery of a wide
range of documents and paragraph formats under perspective transformation in a
single image, without knowledge of the focal length of the camera.  Projection
profiles from hypothesised vanishing points are used to robustly recover the
horizontal vanishing point of the text plane, and to segment the paragraph into
its constituent lines.  Line fitting on the margins and central line of the
document is then applied to deduce the formatting style of the paragraphs.  To
estimate the vertical vanishing point, for fully justified paragraphs the margin
lines are intersected.  For other styles of documents, the observed difference
in the spacings of the lines of text are used to retrieve the tilt of the text
plane, and hence the vertical vanishing point.  Using the two principal
vanishing points we find the orientation of the text plane and recover a
fronto-parallel view.  The algorithm performs well for all types of paragraphs,
provided it has some type of formatting. Hence, as a by-product of our proposed
method we have demonstrated how to understand the format of paragraphs which can 
be of many uses in the Document Recognition field.
In total the process takes around 20 seconds to recover a document,
demonstrating its potential and applicability 
%to realtime systems, as well as its suitability
for normal scanning tasks.
%An optimisation algorithm for the search for the horizontal vanishing point was presented.

%Whilst the described method is suitable for documents or single paragraphs, due
%to the use of margins or line spacing to recover the vertical vanishing point,
%there must be at least two lines present for fully justified paragraphs, and at
%least five for other styles of document, to provide enough data for an accurate
%fitting.  However, other forms of text appear in our environment, and it would
%also be desirable to retrieve these, especially in a realtime system for the
%visually impaired.  Examples of such situations include signs and notices, the
%names of shops on a high street, labels on everyday objects, and number plates.
%To retrieve such isolated lines of text from single images, use can be made of
%the fact that strong horizontal and vertical edges are exhibited by characters
%in many typefaces, providing clues to the orientation of the text.
%Alternatively, by using a video camera to track text regions over image
%sequences, the relative orientation of the text plane may be obtained by
%monocular structure-from-motion algorithms, allowing for fronto-parallel OCR as
%proposed in this work.

Although the resulting images reproduced here are at low resolution, most of
them are nevertheless suitable to be fed to an OCR system to interpret the text
or to be read by a human observer. 
In the future we intend to integrate the work described here and
in~\cite{justin} towards an automatic system for text recognition in the
environment, suitable for a wearable computer system. 

% {\bf acknowledgement}
% The authors would like to thank HP Research Labs, UK for their support.

\bibliographystyle{elsart-num}
\bibliography{jrefs}

\end{document}
